\input{header.tex}

\newcommand{\hmwkTitle}{Assignment\ \#3} % Assignment title
\newcommand{\hmwkDueDate}{March\ 20,\ 2016} % Due date
\newcommand{\hmwkClass}{Algorithms} % Course/class
\newcommand{\hmwkAuthorName}{Zhaoyang Li (2014013432)} % Your name
%----------------------------------------------------------------------------------------

\begin{document}

\maketitle

%----------------------------------------------------------------------------------------
%	TABLE OF CONTENTS
%----------------------------------------------------------------------------------------

%\setcounter{tocdepth}{1} % Uncomment this line if you don't want subsections listed in the ToC

\newpage
\tableofcontents
\newpage

%----------------------------------------------------------------------------------------
%	PROBLEM 1
%----------------------------------------------------------------------------------------


\begin{homeworkProblem}

CRLS Ex. 8.4-4.

\problemAnswer{
This problem is all about defining the buckets, to make sure the expected numbers of points in each buckets $n_i$ are the same.

Choose $r_i$ such that $r_i = \sqrt{\frac{i}{n}}$, $i=0,1,2,...n$, meaning that

$$\forall i, \pi r_i^2- \pi r_{i-1}^2= \pi \frac1n$$

$p_j$ falls into bucket $i$, if and only if $r_{i-1} \leq d_j < r_i$, $i,j \in \{1, 2, ..., n\}$.

It's obvious that the process of placing all the points into the buckets takes at most linear time.

Define indicator random variables $X_{ij} = I\{p_j\ \mbox{falls in bucket}\ i\}$.

Therefore, $n_i = \sum_{j=1}^n X_{ij}$. Applying geometric probability model, we have $E\left( x_{ij}\right) = \frac1n, E\left( x_{ij}x_{ik}\right) =\frac1{n^2}$.
\[
\begin{aligned}
E\left( n_i^2 \right) 
&= \sum_{j=1}^n E\left(X_{ij}^2\right) + \sum_{1\leq j \leq n} \sum_{1\leq k \leq n, k\neq j}E\left(X_{ij}X_{ik}\right) \\ 
&= \sum_{j=1}^n \frac1n + \sum_{1\leq j \leq n} \sum_{1\leq k \leq n, k\neq j}\frac{1}{n^2} \\ 
&= 2-\frac1n\\ 
\end{aligned}
\]

\[
\begin{aligned}
E\left( T(n) \right) 
&=\Theta(n)+\sum_{i=0}^{n-1}O\left(E\left(n_i^2\right)\right)  \\
&=\Theta(n) + n\cdot O\left( 2-\frac1n \right) \\ 
&=\Theta(n)\\
\end{aligned}
\]

So the algorithm has a linear average running time.
}
\end{homeworkProblem}


%----------------------------------------------------------------------------------------
%	PROBLEM 2
%----------------------------------------------------------------------------------------


\begin{homeworkProblem}
Compare insertion sort, quick sort, merge sort and radix sort on randomly generated 32-bit unsigned integers.\\

\problemAnswer{
Sorting algorithms are implemented in C++.

Correctness is verified by checking consistency of the results of the four algorithms.

Notes on generating large random integers: std::rand() gives 15-bit random integers, splicing three of which gives 32-bit random integers.

Execute \url{bin/SortAlgorithms.exe} to get a table like what is shown below.
 \\
\begin{center}

CPU Time Consumed\footnote{Compilation: MSVC++ 11.0 x64 Release; runs on an Intel Core i5 4120U Processor, 4GB DDR3 Memory, Windows 10 Pro.} (microseconds)

\begin{tabular}{rrrrr}
\hline
Size of problem& Insertion& Merge& Quick& Radix \\
\hline
 10 & 0.0717057 & 0.347331 & 0.640546 & 1.64938 \\
 100 & 3.53864 & 2.32687 & 5.57795 & 9.32336 \\
 1000 & 179.396 & 71.1268 & 66.0131 & 84.4482 \\
 10000 & 16285.7 & 918.182 & 1052.08 & 863.248 \\
 100000 & 1.78e6 & 11222.2 & 11777.8 & 8500 \\
 1000000 & 1.71e8 & 143000 & 134000 & 83000 \\
 10000000 & \text{} & 1.62e6 & 1.5e6 & 834000 \\
 100000000 & \text{} & 1.92e7 & 1.67e7 & 8.85e6 \\
 200000000 & \text{} & 3.99e7 & 3.64e7 & 1.9e7 \\
\hline
\end{tabular}

\end{center}
A double-logarithmic chart visualizing the data above:
\begin{center}
\includegraphics[width=0.75\columnwidth]{chart}
\end{center}

It's clear that insertion sort has the greatest time complexity $\Theta(n^2)$, while radix sort has the least $\Theta(n)$.

Quick sort and merge sort are in between, and are basically the same. As size of the array grows, quick sort has the shorter running time.
}
\end{homeworkProblem}

%----------------------------------------------------------------------------------------
%	PROBLEM 3
%----------------------------------------------------------------------------------------


\begin{homeworkProblem}
Give an $O(n \lg n)$-time algorithm to find the longest monotonically increasing subsequence of a sequence of n numbers.

\problemAnswer{

Combining dynamic programming and binary search techniques gives $O(n \lg n)$ time complexity.

The algorithm is implemented in C\#. Execute \url{bin/LongestIncreasingSubsequence.exe} to bring up a GUI.

A run-time screenshot is shown below.
\begin{center}
\includegraphics[width=0.5\columnwidth]{screenshot}
\end{center}

Correctness is verified by checking results of test samples by hand. 
}
\end{homeworkProblem}



%----------------------------------------------------------------------------------------

\end{document}
