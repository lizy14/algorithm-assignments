\input{header.tex}

\usepackage{clrscode}

\newcommand{\hmwkTitle}{Assignment\ \#4} % Assignment title
\newcommand{\hmwkDueDate}{March\ 28,\ 2016} % Due date
\newcommand{\hmwkClass}{Algorithms} % Course/class
\newcommand{\hmwkAuthorName}{Zhaoyang Li (2014013432)} % Your name
%----------------------------------------------------------------------------------------

\begin{document}

\maketitle

%----------------------------------------------------------------------------------------
%	TABLE OF CONTENTS
%----------------------------------------------------------------------------------------

\setcounter{tocdepth}{1} % Uncomment this line if you don't want subsections listed in the ToC

\newpage
\tableofcontents
\newpage

%----------------------------------------------------------------------------------------
%	PROBLEM 1
%----------------------------------------------------------------------------------------


\begin{homeworkProblem}

CRLS Problems 15.5: Edit Distance.

\problemAnswer{
\subsection{Algorithm}
Define $d_{i,j}$ as edit distance between prefix substrings x[1...i] and y[1...j]. A substructure can be described as follows:


\begin{equation}
d_{i,j}=\min
\begin{cases}
    d_{i,j-1}   + w_\mathrm{INSERT} & \\
    \begin{cases}
        d_{i-1, j-1} + w_\mathrm{COPY} & \text{if}\ a_{j} = b_{i}\\
        \infty & \text{otherwise}\\
    \end{cases}\\
    \begin{cases}
        d_{i-2, j-2} + w_\mathrm{TWIDDLE} & \text{if}\ b_{i-1}=a_{j-2}\ \text{and}\ b_{i-2}=a_{j-1}\\
        \infty & \text{otherwise}\\
    \end{cases}\\
    d_{i-1, j}  + w_\mathrm{DELETE} & \\
    d_{i-1,j-1}  + w_\mathrm{REPLACE} & \\
\end{cases}
\end{equation}

The $\mathrm{KILL}$ operation is handled elsewhere, by counting number of successive $\mathrm{DELETE}$'s $n$ as suffix to the sequence of operations. If $n w_\mathrm{DELETE} \geq w_\mathrm{KILL}$, replace the $\mathrm{DELETE}$'s with a $\mathrm{KILL}$.
}
\problemAnswer{

\subsection{Implementation}
Implemented using bottom-up dynamic programming, in $\Theta(mn)$ time where $m,n$ are length of the two strings, in the C++ Programming Language, in the integrated development environment Visual Studio 2012, on the operating system Windows 10.

Correctness is verified by: running against carefully designed test samples, such that the strings are short and the expected results are obvious; running against complicated input, and comparing my implementation's output with my roommate's.
}
\problemAnswer{

\subsection{Testing via GUI}
A graphic user-interface is built with Qt 5. Execute \url{bin/EditDistance.exe} to see the GUI. The sequence of operation to get minimal cost are shown in the form of table. A run-time screenshot is shown below.

\includegraphics[width=0.75\columnwidth]{screenshot}
}
\problemAnswer{
\subsection{As finding an optimal alignment}
Set of operations
$$\{\mathrm{COPY}, \mathrm{INSERT}, \mathrm{DELETE}, \mathrm{REPLACE}\}$$

Corresponding costs
$$\{-1, 1, 1, 2\}$$

Placing a negative sign before the calculated edit distance gives us the score of optimal alignment.
}

\end{homeworkProblem}


%----------------------------------------------------------------------------------------
%	PROBLEM 2
%----------------------------------------------------------------------------------------


\begin{homeworkProblem}

CRLS Exercises 16.2-6: Solve fractional knapsack problem in $O(n)$ time.

\problemAnswer{
To get $O(n)$ time complexity, we have to avoid sorting. Binary search came into my consideration.

The algorithm can be described as follows:

\begin{quote}
0. Calculate $a_i=\frac{v_i}{w_i}, \forall i$.

1. Find median of $\{a_i\}$, denoted as $m$.

2. Create subsets $$G=\{i: a_i>m\}, E=\{i: a_i=m\}, L=\{i:a_i<m\}$$ Calculate $$W_G=\sum_{i\in G}w_i, W_E=\sum_{i\in E}w_i$$

3. Compare $W_G, W_E$ with $W$.

If $W_G>W$, do recursively on $G$;

If $W_G\leq W$, take every item in $G$, and add $i\in E$ as much as possible. Once we have $W_G+W_G\geq W$, we are done;

Do recursively on $L$ with $W' = W-W_G-W_E$.
\end{quote}

That's where we get pseudo-code, shown below.
}


\problemAnswer{
\begin{codebox}
\Procname{$\proc{Fractional-Knapsack-Problem-Aid(w, v, a, b, W)}$}
\li WG = 0, WE = 0, let G, E, L be new arrays
\li (m, n) = \proc{Find-Median}(a, b);
\li \For i = 1 \To b.length
\li    \Do
            \If a[b[i]] $<$ m
\li                \Then L.add(b[i])
\li         \ElseIf a[b[i]]==m
\li                \Then G.add(b[i]), WG += a[b[i]]
\li         \ElseNoIf E.add(b[i]), WE += a[b[i]]
    \End \End
\li \If WG $>$ W
\li \Then \Return \proc{Fractional-Knapsack-Problem-Aid}(w, v, a, G, W)
    \End
\li let R be a new array
\li S = 0
\li \For i = 1 \To G.length
\li     \Do R.add((G[i], 1)); \End
\li \For i = 1 \To E.length
\li    \Do \If WG + a[E[i]] + S $leq$ W
\li        \Then R.add((E[i], max((W - WG - a[E[i]] - S) / w[a[i]], 1)));
\li    \ElseNoIf \Return R
       \End
    \End
\li \Return R + \proc{Fractional-Knapsack-Problem-Aid}(w, v, a, L, W - WG - WE);
\End
\end{codebox}
\begin{codebox}
\Procname{$\proc{Fractional-Knapsack-Problem(n, w, v, W)}$}

\li let a, b be new arrays
\li \For i = 1 \To n
\li \Do		a[i] = v[i] / w[i], b[i]=[i] \End
\li \Return \proc{Fractional-Knapsack-Problem-Aid}(w, v, a, b, W)
\end{codebox}
}

\problemAnswer{
Let $T(n)$ be running time of problem sized $n$. Then
$$T(n) = T(\frac{n}{2})+\Theta(n)$$

Applying master theorem case 3,
$$T(n) = \Theta(n)$$

QED.
}

\end{homeworkProblem}

%----------------------------------------------------------------------------------------
%	PROBLEM 3
%----------------------------------------------------------------------------------------


\begin{homeworkProblem}

CRLS Problems 16-2 Scheduling to minimize average completion time

\problemAnswer{
}

\end{homeworkProblem}


%----------------------------------------------------------------------------------------
%	PROBLEM 4
%----------------------------------------------------------------------------------------


\begin{homeworkProblem}

CRLS Problems 16-5 Off-line caching

\problemAnswer{
}

\end{homeworkProblem}



%----------------------------------------------------------------------------------------

\end{document}
